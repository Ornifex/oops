\documentclass[11pt,a4paper]{article}
\usepackage[english]{babel}
\usepackage{latexsym}
\usepackage{amssymb,amsmath}

\author{Elske van der Vaart \and Gert van Valkenhoef}
\title{MAS Project Proposal: An Automated Logics Toolkit}

\begin{document}

\maketitle

\section*{Problem Statement}

Automated toolkits are very useful tools for users of epistemic logic. They
provide quick non-labor intensive solutions to problems such as consistency,
provability, satisfiability, proof generation, formula simplification or
normalform generation.

The Logics Workbench (LWB) is one such toolkit that allows checking for
provability and consistency in epistemic logic, among others. For this reason
it is used in e.g.  the Multi-Agent Systems course.

However, the LWB is problematic in several respects. Its code base appears to
be outdated and does not support modern Unix environments very well. Several
archaic libraries are required by the LWB distribution. These are often not
available in modern Unix variants such as Linux or MacOS X. 

Furthermore, the usability of the LWB leaves a lot to be desired. Terminal
support for the ASCII version is poor and the X11 interface appears to be
non-functional.

Finally, the LWB does not offer support for the $\textbf{S5}_{(n)}$ system.
For the MAS course, support for this system would be a very useful addition.

However, given the restrictive copyright license, which forbids modification
of the program and the fact that the source code is not publicly available, it
is not feasible to remedy these flaws within the LWB distribution within the
timeframe of a MAS project. Even if we get permission to modify the source
code, the next group wanting to work on a new addition will have to go
through the same process to get permission.

\section*{Project Proposal}

We propose to build a new library that is able to do provability and
consistency checks in $\textbf{S5}_{(n)}$. The library will be implemented in
Java, because of several advantages:

\begin{itemize}
\item Large available class library 
\item Many useful open source third party libraries 
\item Familiarity with the language
\item High portability
\item Easy creation of a web interface (applet)
\end{itemize}

The focus of this project will be on the provability engine. Work on a good
user interface (command line or graphical) will only be done to the extent
necessary to verify that the library works. The program should be in a state
such that it is usable by third parties at the end of the project.

We hope to have full support of $\textbf{S5}_{(n)}$, however if this proves to
be impossible within the available time, a simpler system may be chosen.

It is our hope that this library will provide a basis for further development
either in the form of future MAS projects or final assignments for  some
usability course. Therefore we will select a liberal license for the software
we create.

We do not have an extensive background in automated reasoning, but as this
project appears on the list of possible projects, we assume that it is in
principle possible to do this in the allotted time. We are experienced with
tableau proofs and have written quite extensive programs in Java, though. Gert
is also familiar with parser generation, build systems, unit tests and XML
parsing, among other things, so the peripheral issues don't need to take up an
excessive amount of time.

\end{document}
