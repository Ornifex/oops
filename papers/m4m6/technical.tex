\section{Technical description}

In this section, we will first give a brief description of the tableaux method
used in \oops\ and summarize the formal properties (soundness, completeness,
complexity) of the system. This is followed by a description of how the
tableaux system is implemented in Java. Finally, we describe the input language
for modal formulas that is provided by \oops.

\subsection{\oops\ tableaux}

The \oops\ tableaux system for $S5_n$ is a Java \citep{gosling2005}
implementation of the proof system {\bf ELtap} \citep{deboer2006}.  {\bf
ELtap}, in turn, draws on \citet{fitting1999} and \citet{beckert1997}.
\citet{halpern1992} provide a good review of tableaux methods for modal logics.

FIXME: description/explanation of tableaux, specific rules used by \oops.

The proof system used by \oops\ has been shown to be both sound and complete
\citep{valkenhoef2008}.  Furthermore, in the same work, the implementation was
shown to correspond to the formal description of the proof method.
Unfortunately, this work also shows that the algorithm used by \oops\ is in
{\sc EXPTIME}, whereas satisfiability for $S5_n$ is known to be {\sc PSPACE}
complete \citep{halpern1992}.
However, we believe that for educational purposes the functionality  offered
by \oops\ (see Section~\ref{sec:features}) easily makes up for this
shortcoming. Moreover, the implementation of these features does not depend on
the specific proof rules used. Thus, as future work, the current algorithm may
be replaced by one that is in {\sc PSPACE}.

\subsection{Implementation}

\subsection{Input language}
