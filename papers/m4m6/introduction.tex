\section{Introduction}
When teaching modal logic in a classroom setting, it is helpful to have an automatic theorem prover available. Even a very simple program, that can only answer 'true' or 'false' in response to specific queries, is already useful, as it allows students to check their own work. However, for an {\it ideal} educational proof tool,  a number of other properties seem desirable.

Some of these properties are quite self-evident. For instance, if an automatic theorem prover is to aid students in understanding a logic, it should be capable of displaying its proofs in a readable manner, and it should be able to generate counterexamples to formulas that are false. Furthermore, if it is to be useful in an educational context, it should be easy to install under any operating system, and its user interface should be clear.

Other desirable properties are less obvious, but can nevertheless present themselves in the course of teaching. For instance, one attractive classroom assignment is to have students model riddles, and then derive their outcomes. For the ideal educational proof tool, this suggests two further properties: First, it should allow rich input options, so that students can construct 'theories' from which to draw conclusions. Second, it should feature either $\mathbf\mathit{S}\textbf{5}_{(n)}$ or $\mathbf\mathit{KD}\textbf{45}_{(n)}$ as one of its available logics, as these are the ones typically used to capture human reasoning processes.

As far as we are aware, there are currently no automatic theorem provers that capture all of these aspects. 

 The two systems that appear to come closest are LoTReC [ref]  and the Logics Workbench [ref]. 

LoTReC [ref] is a generic tableau prover, built to provide an easily extensible program suitable for quickly implementing freshly designed logics. However, it also has many of the educational properties listed previously: It displays proof trees and generates counterexamples, it is distributed as a platform independent, double-clickable .jar file, and it comes with a graphical user interface. However, it only allows single formulas as input, and does not feature either $\mathbf\mathit{S}\textbf{5}_{(n)}$ or $\mathbf\mathit{KD}\textbf{45}_{(n)}$. Therefore, one cannot easily model practical problems - such as riddles - in LoTReC [ref].

The Logics Workbench [ref], on the other hand, is explicitly designed as an educational tool, and offers many of the associated functionalities: It can visualize its proofs, it has a graphical user interface, and it allows for the construction of theories. However, at present, it is difficult to install, due to outdated dependencies. There is, for instance, no version of the Logics Workbench that will run under a recent version of the Macintosh operating system. In addition, as the program is not open source, it cannot easily be extended or updated. This also means that the lack of an $\mathbf\mathit{S}\textbf{5}_{(n)}$ or $\mathbf\mathit{KD}\textbf{45}_{(n)}$ module cannot be easily fixed.
