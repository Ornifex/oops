\section{Introduction}
\label{sec:introduction}

In this paper, we describe \oops\footnote{\url{http://wiki.github.com/gertvv/oops}},
an {\it O}bject {\it O}riented {\it P}rover for $S5_n$.
It is a tableau-based theorem prover, aimed at satisfiability checking, that is specifically designed for use in a
classroom setting.
Although not as efficient as highly optimized provers like MSPASS
\citep{mspass2000} and FaCT \citep{fact1998}, or as flexible as highly
extensible provers like LoTReC \citep{lotrec2005} and the Tableau Workbench
\citep{twb2009}, we believe \oops\ offers a unique combination of features
that make it particularly suitable for educational purposes.

First, \oops\ offers support for $S5_n$, the logic typically used to model
human reasoning processes.
As far as we are aware, there are currently no other theorem provers which do
this.
Furthermore, \oops\ is capable of visualizing its tableau proofs, and of
generating counter-models to formulas that are false.
These two properties are very useful for educational proof tools, and they are
shared by at least two other systems: LoTReC can display its tableau trees,
and the Logics Workbench \citep{heuerding1996} can print its derivations in
sequent calculi.

Another feature \oops\ has in common with LoTReC, as well as MSPASS, but with
few other theorem provers, is its free and convenient distribution.
\oops\ is packaged as a ZIP file that includes all dependencies, and once
extracted, can be run simply by double-clicking the resulting oops.jar file.
This is true for any operating system, provided the Java VM is available.
In contrast, the Tableau Workbench must be built from source,
FaCT offers binaries only for Linux and Windows, and requires
Lisp otherwise, and the Logics Workbench is not open source, and can be difficult to install, due to outdated dependencies.

A fifth property that makes \oops\ particularly attractive for educational use
is its Graphical User Interface (GUI).
Although simple, its GUI allows students to input formulas in an easy and
intuitive format, or to save and load files.  LoTReC, MSPASS and the the Logics Workbench also offer GUI-based access.

\oops' final educational attribute is its integrated scripting facility.
In our experience, most student projects involving theorem provers are efforts
to model riddles, or game situations.
This requires that students be able to build and extend theories, using loops
and conditionals where necessary.
Many proof tools, like the Logics Workbench, offer custom scripting languages
for this purpose.
\oops, by contrast, integrates the existing Lua\footnote{\url{http://www.lua.org/}}
scripting language.
Lua can be used to call most of \oops' functions, and offers a rich input
language.

To summarize, our contribution is a proof system, \oops, designed to support
education in logics, specifically in multi-agent reasoning. For this purpose,
\oops\ features a tableau prover for $S5_n$ that is capable of visualizing its
tableau proofs and counter-models for formulas that are false. Furthermore,
\oops\ is platform independent and easy to install. Finally, \oops\ offers a
Graphical User Interface (GUI) and an integrated scripting language (Lua) that
enables easy but powerful interactions with the prover.
 
In the rest of this paper, we first offer a technical description of \oops,
including its proof system, its formal properties, its implementation in Java,
and the details of its input language (Section~\ref{sec:technical}).
This is followed by a more detailed description of \oops' educational
properties (Section~\ref{sec:features}), as well as a worked-out example
(Section~\ref{sec:example}).
We conclude with a discussion of current limitations and future work
(Section~\ref{sec:discussion}).
