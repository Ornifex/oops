\section{Conclusions and Further Work}
\label{sec:discussion}

In this paper, we have presented \oops, a cross-platform, easy to install and
open source tableau prover for $S5_n$.
\oops\ provides users with a graphical user interface, an integrated scripting
language, tableau visualization and counter-model generation. 
We believe these features make \oops\ more suited for educational use than
other similar systems.

Given this, we now identify several directions for further work on \oops.
First of all, although the implementation currently allows new rule sets to be
implemented relatively easily, this requires extending the Java source code and
recompiling \oops. To remedy this, we would like to implement rule sets as Lua
modules and provide such modules for several logics. We would also like to
implement an algorithm that allows the $S5_n$ tableau to be generated in
{\sc pspace}, as our current implementation may require an exponential amount of space.

Furthermore, we would like to have a more complete set of tools to interact
with theories, formulas and Kripke models. For example, it should be possible
to simplify formulas and theories. In the case of Kripke models, we would like
to be able to construct and alter models ourselves and to perform operations
such as model checking and bisimulation. Finally, the GUI should allow users
to provide keyboard input to Lua scripts (through `standard in') so that one
can develop interactive \oops\ scripts.
