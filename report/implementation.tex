{\tt OOPS} is implemented as a Java package (code library) in order to
facilitate reuse in a Logics Workbench-like tool as well as an online applet.
The choice for Java sets {\tt OOPS} apart from similar tools written in Prolog
or C in terms of portability and deployment options.

The package makes extensive use of the new `generics' feature in Java 1.5 in
order to ensure readability and type safety. Most of the code is covered by
Javadoc-style comments that enable automatic generation of API documentation,
which is also available online.

A JUnit 4 testing framework is in place, providing partial testing coverage
for the library. Whenever bugs were encountered, a unit test was added to the
test suite to prevent them from reappearing.

Additionally, several critical parts of the package are covered by an array of
basic tests (although more elaborate tests are desirable) and a set of 25
test formulas (both positive and negative) is available. Tests can be run
automatically by the build tool.

A SableCC based parser for multi modal formulas is included in the package and
a command-line tool that can prove multi modal formulas is provided for
testing and reference purposes. 

Formulas are represented by a collection of objects that implement the {\tt
Formula} interface. Each formula implements functions that check for equality,
perform matching and perform substitution for variables. A formula $\varphi$
is defined to match another formula $\psi$ if a substitution $S$ for the free
variables in $\varphi$ can be found such that $\left[ S \right] \varphi =
\psi$. Formulas and agents can both be variables within a formula, so matching
will return a substitution of formulas for variables and a substitution of
agents for variables.

Labels are defined recursively as a structure containing a world, an agent and
a parent label. All labels implement the {\tt Label} interface. The recursion
is terminated by a {\tt NullLabel} parent label. For any world, the top label
is represented as {\tt NullLabel.Label(world0, NullAgent)}. By definition, no
agent is equal to the {\tt NullAgent}. Using this representation of the top
world, the $k_i = null$ (and similar) conditions for rules do not need to be
implemented explicitly, but follow automatically from the form of the top
label. Labels also have methods to check for equality, perform matching
(returning a substitution for variables) and substitution.

A {\tt Node} in a semantic tableau is a combination of a Label and a Formula.
Nodes can be checked for equality, matched and variables can be substituted
for.

Rules are represented by objects implementing the {\tt Rule} interface. Rules
can be matched to Nodes and contain rewrite rules that result in new Nodes.
There are two types of propositional rules (linear and split) and two types of
modal rules (create - introducing new worlds and access - introducing new
formulas that are necessarily true in a certain set of worlds).

The tableau operates on a queue of matches. On every iteration a new match is
taken from the front of the queue and handled based on the type of rule it
resulted from. For a linear rule, the resultant nodes are placed on the branch
and the matches that can be created from these nodes are put in the queue. For
a split rule, a new tableau is created for each of the resultant nodes. Create
rules create a new world (if this hasn't been attempted yet) and known
necessities are applied to this new world. Access rules introduce new
necessities which are applied to all existing worlds immediately.

A tableau closes if all branches (created by split rules) close. A branch
closes if it closes for any label on the branch.

Infinite recursion is avoided by storing matches from the create rules and
ignoring any duplicates. Non-determinism and backtracking are avoided by
eventually applying all possible rules (when a node matches more that one
rule, all possibilities are explored). The implications for computational
complexity have not been explored in-depth.

The queue used by the tableau is a priority queue, in which rules that are
less `bad' are given higher priority. We hope that this heuristic will prevent
the unnecessary introduction of branches and worlds in many cases, thus
closing the tableau as early as possible. The optimal assignment of badness
values to rules is not known (although a good approximation has been
implemented).
